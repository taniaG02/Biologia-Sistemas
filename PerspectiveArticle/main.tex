\documentclass[nochap]{config/ejercicios}

\title{Predicción de la expresión génica: una revisión}
\author{Tania Gonzalo Santana \\ Claudia Guerrero Rodríguez \\ Sandra Mingo Ramírez \\ Daniel Parra Gutiérrez }
\date{2024/25}

\usepackage[all]{nowidow}
\usepackage{listing}
\usepackage{color}
\usepackage{tabularx}
%\usepackage[sorting=none, backend=bibtex]{biblatex}
%\addbibresource{bibliography.bib}

\begin{document}
\maketitle

Un estudio de Isaacs et al. (2003) muestra cómo la combinación entre métodos experimentales y simulaciones computacionales permiten conocer en profundidad la red de autorregulación del phago lambda.

Contexto:
¿Qué se sabía antes sobre la predicción de la expresión génica a partir de la secuencia?
¿Qué limitaciones existían en los métodos anteriores?
¿Por qué es relevante volver a examinar esta relación?

Contribución del estudio: 
Uso de circuitos genéticos sintéticos con promotores diseñados racionalmente.
Desarrollo de un modelo cuantitativo que relaciona secuencia y expresión.
Prueba experimental en E. coli usando GFP como reportero.

Aspectos técnicos y metodológicos:
Medición sistemática de niveles de fluorescencia para construir el modelo.
Ajuste del modelo con regresión lineal y evaluación de su predictibilidad.
Validación cruzada con nuevos promotores.

Resultados relevantes:
Se puede predecir la expresión con alta precisión en un rango amplio.
Confirmación de que elementos de secuencia contribuyen de manera aditiva a la fuerza del promotor.
Identificación de los sitios más determinantes para la actividad del promotor.

Implicaciones:
Permite diseño predictivo de promotores con fuerza deseada.
Facilita el control preciso de la expresión génica en organismos modificados.
Contribuye al paradigma de diseño de sistemas biológicos con lógica ingenieril.

Limitaciones:
Hecho en un sistema simplificado (E. coli), sin considerar la complejidad de la regulación eucariota.
No se incluyen efectos epigenéticos ni estructuras secundarias del ARN mensajero.
El modelo es lineal y puede no captar todas las interacciones biológicas.

Proyecciones futuras:
Aplicaciones en diseño de terapias génicas o biosensores.
Extensión a otros organismos y redes génicas más complejas.
Integración con aprendizaje automático y modelos más sofisticados.

Reflexión/Crítica:
¿Qué aporta este enfoque a la biología computacional moderna?
¿Qué impacto ha tenido desde su publicación (2003)?
¿Qué desafíos siguen abiertos respecto a la predicción de expresión génica?

\end{document}
