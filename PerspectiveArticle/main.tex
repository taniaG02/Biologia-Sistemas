\documentclass[nochap]{config/ejercicios}

\title{Predicción de la expresión génica: una revisión}
\author{Tania Gonzalo Santana \\ Claudia Guerrero Rodríguez \\ Sandra Mingo Ramírez \\ Daniel Parra Gutiérrez }
\date{2024/25}

\usepackage[all]{nowidow}
\usepackage{listing}
\usepackage{color}
\usepackage{tabularx}
%\usepackage[sorting=none, backend=bibtex]{biblatex}
%\addbibresource{bibliography.bib}

\begin{document}
\maketitle

El estudio de Isaacs et al. (2003) muestra cómo la combinación entre métodos experimentales y simulaciones computacionales permiten conocer en profundidad la red de autorregulación del phago lambda.

%%Contexto:
La mayoría de las funciones celulares resultan de la interacción entre genes, proteínas, RNAs y metabolitos a través de redes de regulación (cita 1). Uno de los grandes desafíos de la biología de sistemas es ser capaz de deducir respuestas fenotípicas celulares a partir de la estructura y comportamiento de estas redes de interacción tan complejas. Estas pueden diseccionarse en módulos, unidades funcionales más sencillas, que facilitan su estudio. 
%¿Qué se sabía antes sobre la predicción de la expresión génica a partir de la secuencia?
%¿Qué limitaciones existían en los métodos anteriores?
%¿Por qué es relevante volver a examinar esta relación?

%%Contribución del estudio: 
%Uso de circuitos genéticos sintéticos con promotores diseñados racionalmente.
%Desarrollo de un modelo cuantitativo que relaciona secuencia y expresión.
%Prueba experimental en E. coli usando GFP como reportero.

Este estudio muestra con un enfoque integrado el funcionamiento de una red génica autorregulada con un modelo cuantitativo y reduciendo la complejidad de las redes génicas a los componentes esenciales. Esta combinación de un enfoque teórico y experimental permite resolver la complejidad de las redes reguladoras de genes a gran escala, analizando los estados de expresión de genes y comprendiendo los patrones globales de expresión y evaluación de la letalidad génica.

%%Aspectos técnicos y metodológicos:
%Medición sistemática de niveles de fluorescencia para construir el modelo.
%Ajuste del modelo con regresión lineal y evaluación de su predictibilidad.
%Validación cruzada con nuevos promotores.

Para los modelos experimentales se realizó una medición sistemática de niveles de GFP para la cuantificación de la expresión. Los resultados experimentales se utilizaron para ajustar un modelo con regresión lineal y evaluar su predictibilidad. 

%%Resultados relevantes:
%Se puede predecir la expresión con alta precisión en un rango amplio.
%Confirmación de que elementos de secuencia contribuyen de manera aditiva a la fuerza del promotor.
%Identificación de los sitios más determinantes para la actividad del promotor.

Este artículo ha demostrado que es posible predecir la expresión génica con una alta precisión en un rango amplio. Mediante modelos teóricos y simulaciones se ha podido confirmar que los elementos de secuencia contribuyen de manera aditiva a la fuerza del promotor, identificando los sitios más determinantes para la actividad del promotor.

%%Implicaciones:
%Permite diseño predictivo de promotores con fuerza deseada.
%Facilita el control preciso de la expresión génica en organismos modificados.
%Contribuye al paradigma de diseño de sistemas biológicos con lógica ingenieril.

%%Limitaciones:
%Hecho en un sistema simplificado (E. coli), sin considerar la complejidad de la regulación eucariota.
%No se incluyen efectos epigenéticos ni estructuras secundarias del ARN mensajero.
%El modelo es lineal y puede no captar todas las interacciones biológicas.

%%Proyecciones futuras:
%Aplicaciones en diseño de terapias génicas o biosensores.
%Extensión a otros organismos y redes génicas más complejas.
%Integración con aprendizaje automático y modelos más sofisticados.

%%Reflexión/Crítica:
%¿Qué aporta este enfoque a la biología computacional moderna?
%¿Qué impacto ha tenido desde su publicación (2003)?
%¿Qué desafíos siguen abiertos respecto a la predicción de expresión génica?

\end{document}



########################## Tania #############
#### Version 1

\documentclass{article}
\usepackage[utf8]{inputenc}
\usepackage{amsmath}
\usepackage{graphicx}

\title{Artículo de perspectiva}
\author{}
\date{}

\begin{document}

\maketitle

\section*{Resumen}
Isaacs et al. aislaron y cuantificaron un módulo de autorregulación positiva extraído del operador derecho (OR) del fago λ en \textit{E. coli}. Mediante un bucle de realimentación positiva en el que la proteína represora λ mutada (cI857) activa su propia transcripción junto a un gen reportero GFP, midieron por citometría de flujo la distribución de fluorescencia celular a distintas temperaturas. Se estableció un modelo determinista, el cual predijo la aparición de bistabilidad (dos estados estables de expresión) al aumentar la tasa de desnaturalización térmica de cI857. No obstante, este modelo también predice histéresis, lo cual no se observó experimentalmente. Esto se resolvió mediante la incorporación de ruido intrínseco (modelo estocástico, Langevin). Este segundo modelo reprodujo cuantitativamente los histogramas, la media y el coeficiente de variación observados experimentalmente.

\section*{Perspectiva y contexto}
\subsection*{Módulos genéticos como “ladrillos” de redes}
Hartwell et al. propusieron que las redes regulatorias pueden entenderse como ensamblajes de “módulos” funcionales (feedback, feed forward, etc.) cuya caracterización aislada facilita la predicción de comportamientos complejos en redes mayores. El trabajo de Isaacs et al. ejemplifica este enfoque “bottom up”: extraer un motivo de realimentación positiva y someterlo a escrutinio cuantitativo, antes de reinsertarlo en arquitecturas mayores.

\subsection*{Bistabilidad y control de ruido}
La biestabilidad (capacidad de un sistema para permanecer en dos estados estables) es algo distintivo de la realimentación positiva. El estudio demuestra no sólo la bifurcación teórica, sino que el “ruido” inherente a bajos números de moléculas es suficiente para inducir transiciones entre estados sin necesidad de fluctuaciones externas. Esto conecta con trabajos de Elowitz \& Leibler (2000) sobre ruido intrínseco/extrínseco en expresión génica, y subraya la función dual del ruido: a menudo considerado perturbador, aquí es un motor de cambio de estado programado.

\subsection*{Modelado híbrido: determinista más estocástico}
La integración de un modelo determinista (bifurcación de puntos fijos) con términos de ruido tipo Langevin permitió reproducir histogramas de fluorescencia, medias y coeficientes de variación sin ajustar artificiosamente un “parámetro de ruido”. Este enfoque es un precedente de las actuales prácticas en biología de sistemas, donde se combinan simulaciones de reacciones discretas (Gillespie) con análisis continuo para capturar fenómenos multiescala.

\subsection*{Implicaciones para diseño de circuitos sintéticos}
En biocomputación y diseño de circuitos sintéticos, disponer de módulos bien caracterizados—conocidos sus parámetros bioquímicos y su respuesta al ruido—es esencial para construir comportamientos lógicos fiables. El autoregulador de Isaacs et al. funciona como un “bit” biológico con dos estados legibles por fluorescencia, sentando bases para switches, memorias y osciladores de nueva generación.

\section*{Retos y futuras direcciones}
\begin{itemize}
    \item Medición dinámica en una sola célula. El artículo predice transiciones rápidas entre estados (Fig. 3A) que el muestreo por citometría (poblacional) no resuelve temporalmente. Técnicas de microfluídica y microscopía de célula viva podrían cuantificar las tasas de conmutación en tiempo real.
    \item Efecto del contexto genómico. Aquí el módulo está en un plásmido de alta copia; su comportamiento podría diferir a baja copia génomica o en presencia de otros factores (p. ej. Cro). Explorar distintos “contextos” cuantitativamente ayudaría a entender la robustez modular.
    \item Ruido extrínseco y acoplamientos múltiples. ¿Cómo se comporta este módulo cuando se conecta a otros—por ejemplo un lazo de realimentación negativa—o bajo variabilidad en el crecimiento celular? Extender el modelo para incluir ruido extrínseco (tasa de transcripción, variaciones de plásmido) ampliaría su aplicabilidad.
\end{itemize}

\section*{Conclusión}
Isaacs et al. ofrecen un caso de estudio paradigmático de cómo aunar teoría y experimento para desentrañar la física de un módulo genético. Más allá de demostrar bistabilidad, revelan el rol constructivo del ruido molecular y proveen un “ladrillo” cuantificado para ingeniería de redes sintéticas. La perspectiva futura pasa por caracterizar dinámicas de conmutación a nivel unicelular, evaluar efectos de contexto y explorar acoplamientos modulares, avanzando hacia el objetivo de diseñar redes de regulación génica con precisión predictiva.

\section*{Referencias clave}
\begin{itemize}
    \item Hartwell LH, et al. Modular cell biology. Nature 402, C47–C51 (1999).
    \item Elowitz MB \& Leibler S. A synthetic oscillatory network of transcriptional regulators. Nature 403, 335–338 (2000).
    \item Siemering KR, et al. Maturation of fluorescent protein mutants. Curr. Biol. 6, 1653–1663 (1996).
\end{itemize}

\end{document}


#### Version 2
\documentclass{article}
\usepackage[utf8]{inputenc}
\usepackage{amsmath}
\usepackage{graphicx}

\title{Artículo de perspectiva 2}
\author{}
\date{}

\begin{document}

\maketitle

\section*{Módulos Autorreguladores y el Futuro del Diseño de Circuitos Genéticos}
Inspirado en Isaacs et al. (PNAS, 2003) y en el marco de la biología de sistemas

\section*{Resumen}
El estudio de Isaacs et al. (2003) sobre un módulo genético autorregulador derivado del bacteriófago λ subraya la importancia de integrar modelos cuantitativos y experimentos para desentrañar la dinámica de redes biológicas. Su enfoque en la biestabilidad inducida por retroalimentación positiva, mediada por la desestabilización térmica del represor cI857, ofrece \textit{insights} críticos sobre cómo el ruido molecular y la arquitectura de circuitos influyen en las decisiones celulares. Esta perspectiva sitúa sus hallazgos en el contexto de la biología sintética y la modularidad de redes, destacando avances, limitaciones y oportunidades futuras.

\section*{Contribuciones clave y contextualización}
\subsection*{Biestabilidad y ruido interno}
Isaacs et al. demostraron que la desestabilización proteica (vía temperatura) activa un régimen biestable en un circuito de retroalimentación positiva, validando predicciones teóricas mediante mediciones estocásticas a nivel de célula única. Este trabajo refuerza la idea de que el ruido intrínseco, lejos de ser un artefacto, es un componente funcional en sistemas biológicos, capaz de impulsar transiciones entre estados estables (ej., diferenciación celular o patologías como el cáncer).

\subsection*{Modelado Cuantitativo Predictivo}
La integración de parámetros bioquímicos conocidos (afinidad de unión a operadores, tasas de dimerización) en un modelo estocástico permitió una comparación rigurosa con datos experimentales. Este enfoque resalta la necesidad de modelos mecanicistas para predecir comportamientos emergentes en biología sintética.

\subsection*{Modularidad en Redes Biológicas}
Al aislar un módulo autorregulador de una red natural (fago λ), el estudio ilustra cómo la disección de subsistemas facilita su caracterización y posterior reintegración en redes complejas. Esto alinea con visiones modernas de la biología de sistemas, donde la modularidad es clave para entender y diseñar funciones celulares (Kitano, 2002; Cell).

\section*{Limitaciones y Desafíos}
\subsection*{Escalabilidad}
Mientras que módulos aislados son manejables, su acoplamiento en redes multicapa (ej., interacciones metabolismo-transcripción) introduce no linealidades impredecibles.

\subsection*{Contexto Celular}
El estudio usó plásmidos de alto número de copias, pero en sistemas genómicos nativos (copias únicas), el ruido y la dinámica podrían variar significativamente.

\subsection*{Aplicaciones In Vivo}
La dependencia de la temperatura como interruptor limita su uso en organismos homeotérmicos. Alternativas (moléculas pequeñas, luz) requieren exploración.

\section*{Direcciones Futuras}
\subsection*{Ingeniería de Circuitos Robustos}
Diseñar módulos autorreguladores que amortigüen o amplifiquen ruido según el contexto (ej., biosensores o terapias celulares).

\subsection*{Integración con Otras Plataformas}
Combinar retroalimentación positiva/negativa y acoplar módulos a osciladores (ej., relojes circadianos sintéticos).

\subsection*{Herramientas de Modelado Multiescala}
Desarrollar algoritmos que conecten dinámicas estocásticas a nivel molecular con comportamientos tisulares.

\section*{Figuras Sugeridas}
\begin{itemize}
    \item Figura 1: Arquitectura del Módulo Autorregulador y Bistabilidad
    \begin{itemize}
        \item Esquema del circuito de retroalimentación positiva (gen cI857, promotor P₆M, GFP).
        \item Gráfico de bifurcación mostrando estados estables (alto/bajo) en función de γₓ (tasa de desestabilización).
        \item Comparación de histogramas experimentales vs. simulaciones estocásticas.
    \end{itemize}
    \item Figura 2: Ruido como Herramienta Funcional
    \begin{itemize}
        \item Diagrama conceptual del ruido intrínseco (fluctuaciones en número de moléculas) vs. extrínseco (variabilidad celular).
        \item Ejemplos biológicos: diferenciación en \textit{Bacillus subtilis}, transiciones metastáticas en cáncer.
    \end{itemize}
    \item Figura 3: Modularidad en Biología de Sistemas
    \begin{itemize}
        \item Red natural del fago λ vs. módulo aislado.
        \item Visión jerárquica: módulos → redes → funciones celulares (ej., motivos de red como retroalimentación, feed-forward).
    \end{itemize}
    \item Figura 4: Aplicaciones en Biología Sintética
    \begin{itemize}
        \item Diseño de interruptores térmicos/ópticos para controlar expresión génica.
        \item Circuitos autorreguladores en terapia celular (ej., liberación de fármacos bajo umbrales de señal).
    \end{itemize}
\end{itemize}

\section*{Conclusión}
El trabajo de Isaacs et al. es un hito en la caracterización cuantitativa de módulos genéticos. Al vincular modelos estocásticos con datos experimentales, no solo valida principios teóricos, sino que también ofrece un marco para diseñar circuitos sintéticos con funciones predecibles. Futuros avances dependerán de superar desafíos de escalabilidad y contexto, aprovechando herramientas emergentes en edición genética y biocomputación.

\end{document}

######################## Claudia ###########################################
🧾 Título sugerido:
"Ruido, regulación y decisión: una nueva visión de la expresión génica desde la dinámica estocástica"

🧠 Artículo de perspectiva
Introducción

Durante décadas, los modelos deterministas han sido la piedra angular para entender la regulación génica. Sin embargo, el trabajo seminal de Isaacs et al. (2003) marca un punto de inflexión al mostrar que, incluso en sistemas regulados con precisión como los circuitos de retroalimentación positiva, el ruido molecular puede dominar el comportamiento dinámico celular. Su estudio, que combina elegantemente experimentos con un circuito sintético y simulaciones estocásticas, demuestra que las fluctuaciones estocásticas pueden no solo perturbar, sino reescribir los resultados predichos por modelos clásicos.

Un sistema sintético como banco de pruebas

Isaacs y colaboradores construyen un sistema de regulación génica sintético en E. coli, basado en un represor mutante del gen cI del fago λ, capaz de autorregular su expresión. Este diseño simple pero potente les permite explorar directamente la dinámica de la retroalimentación positiva, un motivo central en muchos procesos biológicos como la diferenciación celular o la toma de decisiones binarias.

Modulando la temperatura, controlan la degradación de las proteínas y, por tanto, la fuerza efectiva del circuito de autorregulación. Esto les permite explorar un amplio rango de comportamientos, desde estados monoesables hasta regiones donde, en teoría, debería emerger una clara histéresis entre estados de alta y baja expresión.

El papel inesperado del ruido

Lo sorprendente no es solo que el sistema no muestre la histéresis predicha por el modelo determinista, sino que las simulaciones estocásticas sí capturan este fenómeno. Las transiciones entre estados no se deben a cambios estructurales, sino a fluctuaciones en el número absoluto de moléculas debido al carácter aleatorio de los eventos bioquímicos en células individuales.

Este hallazgo reconfigura la manera en que entendemos la estabilidad en biología molecular. Un sistema puede ser teóricamente bistable, pero si las barreras energéticas son comparables con la magnitud de las fluctuaciones, esas barreras se desdibujan. Así, el comportamiento observado en una población es el resultado de muchas trayectorias ruidosas, no de un paisaje determinista.

De la precisión a la plasticidad

Este estudio sugiere que la biología no busca siempre minimizar el ruido, sino que puede aprovecharlo como una fuente de flexibilidad funcional. En lugar de ser un simple "error de la máquina", el ruido puede facilitar decisiones celulares, permitir respuestas adaptativas, e incluso crear heterogeneidad que favorezca la supervivencia en entornos cambiantes.

Isaacs et al. logran así un cambio conceptual: no todo lo que parece aleatorio es desordenado; lo estocástico puede ser biológicamente significativo.

Perspectivas futuras

Este trabajo abrió el camino para una explosión de estudios sobre la dinámica estocástica en biología, inspirando nuevas formas de modelar, medir y diseñar sistemas genéticos. Hoy, la ingeniería de circuitos sintéticos incluye el ruido como un parámetro más del diseño, y se reconocen funciones funcionales del ruido en desarrollo, diferenciación y resistencia a tratamientos.

En retrospectiva, este artículo anticipó una visión más rica y realista de la biología molecular, donde el determinismo se mezcla con la probabilidad para generar comportamiento complejo, diverso y evolutivamente robusto.

Conclusión

La contribución de Isaacs et al. va más allá de un sistema experimental exitoso: ofrece una lección fundamental sobre cómo pensar la biología.
En el mundo molecular, el ruido no es una falla: es una propiedad del sistema. Y en esa aparente imperfección, reside parte de la riqueza de lo vivo.


Limitaciones:
-  El artículo sugiere que el ruido en la expresión génica es una causa principal de variabilidad, pero no cuantifica completamente las distintas fuentes de ese ruido (transcripcional, traduccional, partición celular…) ->  No se puede asignar peso a cada fuente de fluctuación, lo que complica refinar modelos más precisos.
- 