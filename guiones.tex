%%%%%%%%%%%%%%%%%%%%%%%%%%%%%%%%%%%%%%%%%%%%%Parte Sandra
El funcionamiento celular depende de interacciones complejas entre genes, proteínas, mRNAs y metabolitos, tratándose de una red compleja. De ella se pueden extraer módulos funcionales más pequeños que se pueden estudiar mejor y combinar posteriormente para ampliar la complejidad del sistema. 
Con esta premisa, el artículo escoge un módulo genético autorregulado del phago lambda de una forma integral, combinando métodos experimentales con modelos teóricos para poder crear así un modelo cuantitativo.

La red de autorregulación del phago lambda controla el ciclo de vida viral regulando la expresión de los genes cI y cro. El aislamiento del OR y cI genera la autorregulación.

El módulo escogido por el artículo es el que se ve aquí (fig 1A). El promotor está dividido en tres operadores y a continuación se encuentra el gen cI. El producto es una proteína que dimeriza. Cuando se une al primer sitio de unión en el operón, no se cambia la expresión de la proteína. Al unirse al segundo, la expresión aumenta, pero cuando se une al tercero y el promotor está completamente bloqueado, no hay expresión.

Para estudiar esto se generaron tres plásmidos principales. Se utilizaron dos genes, cI wild-type termoestable y cI857 termosensible, aunque la diferencia entre ambos se debe seguir explorando.

El plásmido pTOR2G sirve como control, no tienen ningún gen cI, sólo gfpmut3. Este gen expresa GFP y permite así determinar la tasa de expresión basal del promotor, sirviendo como medida cuantitativa experimental.

El plásmido pT202b es otro control que tiene tanto cI wild-type como gfpmut3, siendo así termoestable. 

El plásmido pT2002b es igual que el anterior, pero con el gen cI857 termosensible.

%Plásmido pMTRCG: control positivo, promotor constitutivo que expresa gfpmut3

Cambiando la temperatura, se ajustó la estabilidad de la proteína represora y variar así el grado de activación para examinar la dinámica de expresión del feedback loop positivo. 
Con pT2002b, según sube la temperatura, con 39º el represor se desestabiliza, mostrando así dos poblaciones estables (biestabilidad) hasta llegar a los 40º, donde se vuelve a una sola población.
Mismo experimento con pT202b, al tener cI wild-type no termosensible, se mantuvo en un estado monoestable a lo largo de todo el rango de temperaturas. Esto indica que la desestabilización del represor es el responsable de la biestabilidad. 


Parte Determinista (Tania)
- Explicar las reacciones quimicas -> como se relaciona con el modeo de fago lambda que se usa, asunciones
- Modelo determinista: explicar las ecuacionies, produccion, degradación.
- Parámetros
- Análisis del modelo determinista: estudio de puntos de equilibrio con diagrama de fases --> complejo - simplificación
- Análisis del modelo determinista simpificado (1 variable, x): diagrama fases --> biestabilidad
- Biestabilidad en función de condiciones inciales
- Biestabilidad en función de los parámetro
- Histéresis --> si predicho por el modelo

############################################
##########Parte Tania completo ##############
#############################################

Como ha mencionado Sandra, el módulo estudiado se extrajo del fago lambda específicamente del operador derecho (OR), que regula los genes cI (represor) y cro.
Este sistema incluye las siguiente variables clave. X que representa los monómeros del represor cI; X2 que representa los dímeros activos de cI y Di que representa los sitios de unión en el ADN con i dímeros unidos. Por ejemplo:
o	D1: Un dímero unido a OR1.
o	D2: Dos dímeros unidos a OR1 y OR2.
o	D3: Tres dímeros unidos a OR1, OR2 y OR3 (represión).
Con ello, podemos plantear las reacciones químicas de dimerización, unión de dímeros a sus sitios correspondientes en el DNA, las reacciones de trasncripción, traducción y degradación como vemos indicadas en la diapositiva.
---
Con esto, podríamos escribir las ecuaciones diferenciales, establecer las ligaduras como la dominancia del represor cI, donde solo se considera la autorregulacion por cI, ignorando otros reguladores como la proteína Crc del fago lambda, la ausencia de cooperativdad no canónica, es decir, que la unión de los dímeros a los operadores OR1-OR sigue las reglas de afinidad predefinidas (sigma1 y sigma2) y la degradación proporcional donde la tasa de degradación es lineal a la concentración. 
Respecto a las asunciones biológicas, se asume la separación de escala temporales donde tenemos reacciones rápidas (dimerización y unión a DNA) y reacciones lentas (transcripción y degradación) y degradación.
A partir de aquí podemos escribir las ecuaciones diferenciales, despejarlas y obtener el modelo matemático como hemos hecho en las clases impartidas en esta asignatura. 
--
Asi, las ecuaciones finales obtenidas para el modelo que denominan deterministas son las siguientes, donde la x denota el número de monómeros cI, mientras que g se corresponde con el número de moléculas GFP, nuestro reportero (mencionar v?). 
Si nos centramos en la ecuación correspondiente a x, el represor, podemos observar los términos de producción betaf(x,v), que depende de x (retroalimentación positiva). Esta función viene definida de la forma que vemos ahí y, aunque parece compleja, es lo mismo que hemos explicado ya de forma conceptual. Esta función (f(x,v)) captura la activación/represión trasncripcional (depende de la unión de los dímeros a los sitios de unión en el DNA (OR)). Vemos que la función no es lineal debido a la competencia entre activación y represión. Veremos que esto genera un comportamiento biestable, característico de los bucles de retroalimentación positiva.

En el numerador tenemos los términos que se corresponden con la activación transcripcional (es decir, cuando el dímero de cI se une a OR1 y OR2). En el denominador además incluimos los términos de represión (unión a OR3) (los términos de denominador por orden representan la tasa basal, la unión a OR1, la unión a simultánea a OR1 y OR2, la simultánea a los tres sitios OR1, 2 y 3 (represión)).
Por otro lado, tenemos la degradación que es proporcional a la concentración de cI (termino de degradación es sensible a temperatura).
La función h(x,v) permite capturar el efecto de las reacciones rápidas (dimerización y unión al DNA). (explicar cada término)
---
Parámetros principales y su impacto:
Y como podemos observar tenemos muchos parámetros,
•	m: Número de copias del plásmido (A mayor m, mayor producción basal).
•	α: Activación transcripcional por unión a OR2. Factor de activación por OR2. Amplifica la transcripción (α≈11)
•	σ1,σ2: Afinidades relativas de unión a OR2 y OR3. Controlan competencia entre activación y represión.
•	γx : Tasa de degradación de cI. Aumenta con la temperatura; determina transición entre estados.
Dado que nuestro enfoque está en un sistema autorregulatorio derivado del fago lambda, la mayoría de los parámetros son conocidos, lo que permite desarrollar un modelo cuantitativo.

%%Valores Usados:
%%σ1=2, σ2=0.08, α=11, m=60 (copias/plásmido).
%%s1≈2, s2≈0.08, c1≈0.05, c2≈0.33.

La mayoría de los parámetros son conocidos
La tasa de desestabilización (γx) y el número de copias por plásmido (m) determinan el número de moléculas represoras en estado estacionario.
---
Una vez entendidos los términos de las ecuaciones, podemos llevar a cabo un análisis del modelo. Para ello, lo primero que tendríamos que hacer es calcular los puntos de equilibrio, para ello igualamos la ecuación a 0. Si dibujamos el diagrama de fases, este mostraría biestabilidad para un rango determinado de la tasa de degradación. 
Para poder jugar con esto y hacer simulaciones/representaciones en python, tendríamos que resolver la matriz Jacobiana, lo cual es complejo debido a la no linealidad. Por ello, decimos simplificar las ecuaciones, asumiendo un volumen constante e igual a 1. De esta forma ignoramos el crecimiento celular y nos enfocamos en x, el represor como variable principal.
Las ecuaciones quedarían como se muestran en la diapositiva. 
---
De esta forma podemos estudiar más fácilmente la dinámica de la ecuación. Asi, hemos implementado estas ecuaciones simplificadas en Python para graficar el diagrama de fases en función de la tasa de degradación de x, el represor. Este parámetro es dependiente de la temperatura. Asi, a más temperatura, mayor tasa de degradación y cómo podemos observar, bajos valores presentan un único estado estable, valores intermedios (3000-4000) presentan dos estados estables (biestabilidad) y altos valores llevan de nuevo a un único estado estable. 
Esto mismo podemos observarlo si resolvemos las ecuaciones diferenciales y graficamos la función a lo largo del tiempo. Aquí de nuevo vemos que a bajos valores o altos valores del parámetro (tasa de degradación), tenemos un único estado estable en el cual siempre vamos a acabar independientemente de las condiciones iniciales. Sin embargo, en el rango intermedio, donde teníamos biestabilidad, vemos que, en función de las condiciones iniciales, acabamos en un estado con alta concentración de cI o en un estado con baja concentración de cI.
En el artículo el único parámetro que modifican es este que estamos viendo, la tasa de degradación (al variar la temperatura). No obstante, el otro parámetro que podíamos modificarse era m, el número de copias de cI por plásmido. Este parámetro lo establecen en un valor de 60 (las condiciones del experimento). Esto es relevante porque la biestabilidad puede estar presente o no en función del valor de este parámetro. Asi en el artículo establecen m=60, que, como podemos observar en ese rango hay biestabilidad. Un bajo número de copias o un alto número de copias lleva a un único estado estable. 
---
Por tanto, hemos visto que el modelo determinista presenta biestabilidad para un rango concreto del parámtro de la tasa de degradación y además, este modelo presenta histéresis, es decir, qu la trayectoria es distinta al aumentar/disminuir la temperatura (modificar la tasa de degradación). 
Si dibujásemos el diagrama de bifurcación, obtendríamos un punto crítico del tipo silla-nodo.
Esto de forma biológica nos ofrece “memoria”, es decir, no hay una respuesta inmediata, sino que depende de la historia previa. No obstante, esta histéresis no se observará de forma experimental como veremos más adelante por lo que es necesario plantear un nuevo modelo donde se introduce ruido estocástico.



Parte Estocástica (Dani)

- Fórmula (explicar ruido y por qué se mete)
Estos modelos incorporan elementos de aleatoriedad, a diferencia de los modelos deterministas que son completamente predecibles. [En el contexto de la regulación genética, esto probablemente se refiere a las fluctuaciones aleatorias en los niveles de proteínas o ARN debido a eventos moleculares discretos. ]
Con el objetivo de que se puedan observar grandes fluctuaciones y la dinámica de expresión es necesario que exista ruido [de ese modo se podrán captar la esencia del módulo autorregulador.]
La magnitud del ruido depende de las tasas de producción y degradación (dentro de la raíz cuadrada) (a mayor producción (βf), mayor ruido).
ξ(t) es Ruido blanco, un término matemático que representa fluctuaciones completamente aleatorias. Sus características son: 
* Media cero: [No hay sesgo (las fluctuaciones son] igualmente probables hacia arriba o abajo). 
* Varianza unitaria: [La intensidad del ruido] está normalizada. 
* Sin correlación temporal: El valor de ξ(t) en un instante [no depende de valores pasados. Esto simplifica el modelo,] asumiendo que las reacciones ocurren en escalas de tiempo muy rápidas.

Por ese motivo se utiliza la variación aleatoria en el tiempo (xi) de los eventos individuales de reacción bioquímica; lo cual se expresa con esta fórmula y puede simplificarse en estas dos (18 y 19).
[Cabe recalcar que el ruido no es un error, sino una propiedad inherente de sistemas con pocas moléculas. Su inclusión en modelos es crucial para predecir comportamientos realistas, como la variabilidad entre células individuales.]


Parte Claudia: Comparación modelo-experimento + Conclusiones
- A continuación, pasamos a comprar los resultados obtenidos de manera experimental con los del modelo para comprobar si este es capaz de capturar realmente lo que está ocurriendo en las células.
- Comparar la expresión génica del represor lambda en función de la temperatura; la temperatura se introduce en el modelo mediante la tasa de desestabilización del represor lambda. 
	Los histogramas de color verde muestran los resultados experimentales donde la expresión del represor lambda se mide a través de la fluorescencia de GFP. De manera que a mayor represor lambda, mayor GFP y mayor fluorescencia.
	Los histogramas de color azul muestran los valores de GFP y de represor lambda predichos por el modelo.
	Observamos que los resultados son prácticamente idénticos. 
	Además, cuando aparece la biestabilidad vemos que la distribución de ambas poblaciones se solapa indicando que la transición entre estados ocurre de manera muy rápida. (Figura 3) Esto lo predice el modelo: la transición de una célula de un estado a otro ocurre en escalas de tiempo menores que la división celular.
- Valores medios de GFP experimentales y teóricos, observamos que su distribución es prácticamente idéntica. 
	Contar que existen 3 zonas: estado alto de CI, biestabilidad y estado bajo de cI.
	Al inicio del experimento, vemos un pequeño aumento de la concentración de GFP. En el estado de equilibrio, la concentración [GFP] =  γ_x/(γ_g  ) [cI], de manera que, al principio del experimento, al aumentar la temperatura la degradación de la proteína cI se acelera, lo que alivia su efecto represivo sobre el promotor y como resultado, aumenta su transcripción en la que se produce tanto represor lambda como GFP, y como GFP se degrada más lentamente, su concentración se acumula de forma visible. Sin embargo, al continuar aumentando la temperatura, la degradación de CI se vuelve tan intensa que ya no hay suficiente CI para mantener activado el promotor.
El sistema pierde la retroalimentación positiva, se reduce drásticamente la transcripción, y por lo tanto la expresión de GFP también disminuye.
	Quisieron comprobar si el sistema presentaba histéresis, para ello repitieron el experimento al revés de 43-36ºC y observaron que se producía el régimen de biestabilidad a la misma temperatura que en el experimento original, lo que demostraba que no había histéresis. Esto se debe a que el propio ruido es tan fuerte que es capaz de hacer que las células cambien de un estado a otro en escalas de tiempo inferiores al tiempo de incubación para cada temperatura. Además, el modelo determinista sí que predecía histéresis en este sistema, lo que indica la importancia que tiene el ruido en la regulación génica y en la transición de un estado a otro.
- Además, quisieron comparar de manera cuantitativa la diferencia en la expresión génica de GFP, para ello, utilizaron el coeficiente de variación (CV). Las gráficas obtenidas en cada caso son prácticamente iguales y tienen forma de S. 
	Zona estado-alto donde CV es bajo: muchas moléculas en el estado alto 
	Zona de biestabilidad: CV empieza a aumentar, tenemos una parte de la población en el estado alto y la otra en el bajo. De modo que el CV en la zona de transición se correlaciona con una media ponderada de dos estados.
	Zona estado-bajo: el CV alcanza su valor máximo y comienza a disminuir, aunque mantiene unos valores superiores a los de la primera zona. En el estado bajo existe un menor número de moléculas de manera que el ruido genera una mayor dispersión, que se traduce en un mayor CV.


Parte Dani:

Entre nuestras pruebas conseguimos
- Simulación 33: Un modelo que coincide con la figura 3B, en el cual, como ha explicado Claudia previamente, se puede observar también que dentro de un ciclo celular, a escalas de tiempo menores ocurren transiciones de células de un estado a otro.
    [γx es la tasa de desestabilización de la proteína CI, que está relacionada con la temperatura. En el modelo estocástico, γx influye en la dinámica de las fluctuaciones de la expresión génica (por ello se exploranlas fluctuaciones de  fluorescencia de GFP a un valor fijo).]
- Simulación 49: Intentamos obtener un modelo que reflejase la figura 4B pero que también permitiese comparar entre los modelos determinista y el estocástico. A pesar de que no hayamos podido conseguir que se vea reflejada la histeresis en el determinista ni la biestabilidad en el estocástico (ni una población en estado bajo ni otra en el alto), puede observarse esa tendencia a una menor concentración de GFP a mayor temperatura, es decir sí que hemos podido representar su influencia en la expresión génica y su diferente concentración en función del aumento o disminución de la temperatura. 
Lo ideal/el objetivo habría sido que se hubiese podido observar las líneas deterministas (azules) divergiendo al subir y bajar, mostrando histéresis, y las estocásticas (verdes) superponiéndose más, indicando menor histéresis. 
Las limitaciones de nuestro código son:
- Las simplificaciones: 
    - No considera dilución por división celular (clave en el artículo) (seconsideró constante). 
    - Asume GFP instantáneamente fluorescente (ignora tiempo de maduración). 
- Escala de tiempo: Las simulaciones usan 50 unidades de tiempo, pero no se especifica su equivalencia a divisiones celulares (por lo que realmente nos hemos basado en el cambio de temperatura)
Conclusiones (retocadas)
    (En las simulaciones)
    - Con ello podemos simular transiciones rápidas entre estados, relevantes en procesos como diferenciación celular o cáncer; además de una correcta tendencia en la concentración de GFP al aumentar la temperatura.
    - También hemos podido confirmar la existencia de dos puntos de equilibrio estables para gamma_x = 3750 y 3800.
    (En el artículo)
    - En definitiva, es necesario incluir ruido en el modelo para poder representar de manera precisa el comportamiento del módulo autorregulador. [En este caso, se ha podido comprobar tanto de manera teórica y experimental como fluctuaciones en el número de moléculas, debidas a la naturaleza aleatoria de las reacciones químicas, son las responsables de los cambios en la expresión génica y de las transiciones entre estados. ]
    - Es importante evaluar que los estados sean estables y sus implicaciones en el fenotipo; integrando modelos matemáticos y experimentales para simplificar la red compleja. Hay que tener en cuenta que el modelo matemático puede predecir pero es necesario validarlo experimentalmente.
    - El uso de un enfoque "de abajo hacia arriba" (bottom up) que además combina enfoques teórico y experimental puede ser una herramienta valiosa para comprender y resolver la complejidad las redes reguladoras de genes a gran escala.


###################################################
##########ORGANIZACIÓN (propuesto pot Tania)#######
###################################################

1. Introducción (Sandra)
Contexto (puntos a tratar)
•	Las células regulan funciones mediante redes complejas de genes, proteínas y otros componentes.
•	Estas redes pueden dividirse en módulos funcionales (ej.: retroalimentación positiva/negativa).
•	Bistabilidad: Capacidad de un sistema de existir en dos estados estables bajo las mismas condiciones.
•	Ej en el que se basan: El fago lambda usa retroalimentación positiva para decidir entre lisis (destruir célula) y lisogenia (integrarse en el ADN).

Objetivo:
- Demostrar experimentalmente la bistabilidad en un módulo autorregulador aislado del fago lambda.
- Validar un modelo matemático que incorpora ruido interno en la expresión génica.

2. Sistema Experimental (Sandra)
Diseño del módulo genético:
Genes clave:
- cI857: Codifica un represor termo-sensible del fago lambda.
- GFP: Proteína fluorescente para medir expresión génica.
- Promotor Pₘ: Regula la expresión de cI857 y GFP.

Mecanismo:
- A bajas temperaturas, el represor cI857 es estable, activando su propia expresión (retroalimentación positiva).
- Al aumentar la temperatura, el represor se desestabiliza, reduciendo su expresión.

Controles: Plásmidos con cI silvestre (no termo-sensible) para confirmar que la bistabilidad depende de la desestabilización térmica.

-----------------------------------------------------------------------

3. Modelo Matemático (Tania/Dani)

Modelo determinista:
Mini-explicación de la deducción?
Poner ecuaciones
Explicar:
-	Autorregulación positiva
-	Terminios producción/degradación
-	Dinámica de represores (cI) y GFP – biestabilidad, histéresis.
-	Explicar parámetros – lo que significan (no linealidad)
No linealidad: La función f(x,v)f(x,v) modela la activación/represión del promotor Pₘ al unirse represores a sitios operadores (OR1, OR2, OR3).

Modelo estocástico:
-	Añadimos ruido – decir porqué. 
-	Como queda el modelo – biestabilidad, pero no histéresis (lo que cuadra con lo experimental)

Simulaciones (lo que hayamos conseguido)
Biestabilidad: El modelo predice dos estados estables (alto/bajo) para valores intermedios de γx – ¿?? No lo hemos coneguido
Grafica que tenemos

-----------------------------------------------------------------------

4. Resultados Experimentales (Claudia)
Evidencia de biestabilidad:
- Distribuciones bimodales: A 39–40°C, se observan dos poblaciones de células (alta/baja expresión de GFP) (Fig. 1C y 2A).
- Coeficiente de variación (Cv): Aumenta en la zona de transición, reflejando fluctuaciones entre estados (Fig. 4B).

Ausencia de histéresis:
No se observa dependencia de la dirección del barrido térmico, lo que sugiere que el ruido acelera las transiciones entre estados.

Validación del modelo:
Las simulaciones coinciden cuantitativamente con los datos experimentales (Fig. 2B y 2C).

-----------------------------------------------------------------------

5. Discusión/prespectivas futuras/dificultades que hemos encontrado
Importancia del ruido:
- Las fluctuaciones en moléculas pocas (ruido interno) son críticas para la dinámica del sistema.
- Permiten transiciones rápidas entre estados, relevantes en procesos como diferenciación celular o cáncer.

Aplicaciones: ¿?? quitar
- Ingeniería de circuitos genéticos sintéticos con comportamientos predecibles.
- Estudio de redes naturales (ej.: apoptosis, ciclo celular).

6. Conclusiones
•	La integración de modelos matemáticos y experimentos permite diseccionar módulos genéticos complejos.
•	La bistabilidad surge de la retroalimentación positiva y el ruido interno, no solo de parámetros deterministas.
•	Este enfoque "de abajo hacia arriba" es clave para entender redes biológicas a mayor escala.




----------------------------------------------------------------------------
Preguntas para la discusión 
1.	¿Cómo afectaría reducir el número de copias del plásmido a la bistabilidad?
En el artículo, mencionan que usaron plásmidos de alto número de copias (50-70 por célula). Si se reduce el número de copias, probablemente disminuiría la cantidad total de represores cI857 y GFP en la célula. Según el modelo matemático presentado, la síntesis de proteínas depende de parámetros como la tasa de producción basal (β) y el número de copias del plásmido (m). Si m disminuye, la función f(x, v) en las ecuaciones se vería afectada, reduciendo la producción de represores. Esto podría desplazar el sistema hacia un régimen donde la retroalimentación positiva no es suficiente para sostener la bistabilidad, llevando a un estado monostable. Además, al haber menos moléculas, el ruido interno (fluctuaciones estocásticas) aumentaría, lo que podría facilitar transiciones más frecuentes entre estados o incluso eliminar la distinción clara entre ellos. Sería interesante simular esto ajustando el parámetro m en el modelo y observando cómo cambian las distribuciones de expresión.
2.	¿Qué implicaciones tiene la ausencia de histéresis en sistemas biológicos naturales?
La ausencia de histéresis sugiere que el sistema no "recuerda" su estado anterior, permitiendo una transición rápida entre estados según las condiciones actuales. Esto es útil en entornos dinámicos donde la flexibilidad es clave (ej.: respuesta a estrés).
En este estudio, el ruido estocástico dominó sobre efectos deterministas, enmascarando la histéresis. En sistemas naturales más complejos (con múltiples módulos interconectados), la histéresis podría reaparecer, proporcionando robustez frente a fluctuaciones.
El fago lambda silvestre muestra histéresis debido a su red dual (cI y Cro), lo que resalta la importancia del contexto de red en el comportamiento emergente.
3.	¿Podría aplicarse este modelo a otros sistemas de retroalimentación, como los que involucran ARN?
El modelo actual se centra en la regulación por proteínas (represores que se unen a ADN). Para sistemas con ARN (por ejemplo, riboswitches o ARN no codificante que regula expresión génica), habría que adaptar las ecuaciones. En lugar de dimerización de proteínas y unión a operadores, se modelarían interacciones ARN-ARN o ARN-proteína. Además, la dinámica del ARN es más rápida (menor vida media) comparada con las proteínas, lo que podría requerir ajustar los parámetros de degradación (γ). Sin embargo, los principios básicos de retroalimentación y ruido interno seguirían siendo relevantes. Sería importante incluir términos que representen la síntesis y degradación de ARN, y posiblemente competencia por sitios de unión. Simular esto permitiría predecir si la bistabilidad emerge en tales sistemas y cómo el ruido afecta su comportamiento.
